\documentclass[a4paper,12pt,final]{article} %Papierformat, Schriftgröße, BadBox markieren Bilder nich anzeigen (draft/final)
\usepackage[utf8]{inputenc}
\usepackage{lmodern,textcomp} %textfont deutlich runder/schöner + Eurozeichen richtig
%\renewcommand*\familydefault{\sfdefault} %Serifenlos
\usepackage[onehalfspacing]{setspace} %Zeilenabstand
\usepackage[left=2.6cm, right=2.6cm, top=3cm, bottom=3cm]{geometry} %Seitenränder
\usepackage{graphicx} %bilder (includegraphics)
\usepackage{todonotes}
\usepackage[hang]{footmisc} % Fußnoten 2. Zeile einrücken
\usepackage{float} %Positionierung von Bildern etc. [h] 
\usepackage{booktabs} %Tabellen
\usepackage{longtable}
\usepackage{amsmath}
\usepackage{multirow}
\usepackage{siunitx} %Zahlen zb exp oder einheiten
\usepackage{csquotes} %sonst bringt babble n Fehler
\usepackage[backend=biber, %Verarbeitet die BibTex datei und aktualisiert sich mit F11
			style=numeric-comp, %macht nummern aus den Quellen?!?
			sorting=none,]{biblatex} %stell sortiern nach Namen der Autoren aus
\DefineBibliographyStrings{ngerman}{ %macht aus u.a. -> et al.
   andothers = {{et\,al\adddot}},
}
\bibliography{Literatur.bib}  % Literatur verzeichniss
\usepackage[english,ngerman]{babel}  % The last language in the list is the primary one.
\usepackage{pdfpages} %zum Anhängen von PDF's (Reflexion A&B)
\usepackage[pdfusetitle, colorlinks, linkcolor=black, citecolor=black, urlcolor=gray,]{hyperref}  % Add clickable links for figures etc. Comment out for printing if desired.
\numberwithin{equation}{section} %nummeriert Formeln mit Kapitelnummer.Formelnummer ACHTUNG! muss nach hyperref kommen.
\numberwithin{figure}{section} %numeriert Bilder mit Kapitelnummer.Bildnummer ACHTUNG! muss nach hyperref kommen.
\numberwithin{table}{section} %numeriert Tabellen mit Kapitelnummer.Tabelle ACHTUNG! muss nach hyperref kommen.
\setcounter{biburllcpenalty}{9000}% Kleinbuchstaben?? mach die Zeilenumbrüche bei link's
\setcounter{biburlucpenalty}{9000}% Großbuchstaben?? mach die Zeilenumbrüche bei link's
\setcounter{tocdepth}{2} %Tiefe des Literaturverzeichnises, 2=inkl. subsecion so nur eine Seite Inhaltsverzeichnis
\setlength{\parindent}{0pt} %einrücken bei absätzen verhindern
\usepackage{listings}
\usepackage{color}

\definecolor{mygreen}{rgb}{0,0.6,0}
\definecolor{mygray}{rgb}{0.5,0.5,0.5}
\definecolor{mymauve}{rgb}{0.58,0,0.82}
\definecolor{bg_col}{rgb}{1,1,0.7}

\lstset{ %
  backgroundcolor=\color{bg_col},   % choose the background color
  basicstyle=\footnotesize,        % size of fonts used for the code
  breaklines=true,                 % automatic line breaking only at whitespace
  captionpos=b,                    % sets the caption-position to bottom
  commentstyle=\color{mygreen},    % comment style
  escapeinside={\%*}{*)},          % if you want to add LaTeX within your code
  keywordstyle=\color{blue},       % keyword style
  stringstyle=\color{mymauve},     % string literal style
}

\usepackage{fancyhdr} %Kopf/Fußzeile
\pagestyle{fancy}
\setlength{\headheight}{28pt}
%\setlength{\footheight}{30pt}
\lhead{\nouppercase{\leftmark}}
\rhead{\includegraphics[width=3cm]{Bilder/logo.pdf}}
\lfoot{Abt \& Girke}
\cfoot{}
\rfoot{\thepage}
\renewcommand{\headrulewidth}{0.5pt}
\renewcommand{\footrulewidth}{0.5pt}

\begin{document}
\begin{center} %Titelseite
	\begin{figure}[h]
	\begin{center}
		\includegraphics[width=8cm]{Bilder/logo.pdf} %bilder vertical anordnen zueinander
		\vspace{2cm}
	\end{center}
	\end{figure}
	\begin{Large}	
		\textbf{Entwicklung und Umsetzung einer intuitiven Steuerung für eine Roboterhand durch Erfassen
		der Geste einer menschlichen Hand\linebreak \linebreak Kinematik Labor\\}
		\vspace{1.5cm}
	\end{Large}
	\begin{large}
		des Studienganges Mechatronik und Robotik\linebreak		
		an der Frankfurt University of Applied Sciences\linebreak\linebreak
		von
		\begin{longtable}[b]{c c}
		Peter Abt & 1400337\\
		Felix Girke & 1386888\\
		\end{longtable}
		\vspace{1cm}
		\today\linebreak \linebreak
		\begin{longtable}[b]{p{7.9cm} p{6.9cm}}
		Bearbeitungszeitraum: & Wochen\\
		Betreuer & Prof. Dr. Enno Wagner
		\end{longtable}\todo{Bearbeitungszeitraum}
	\end{large}
\end{center} 
\thispagestyle{empty}%ende Titelseite
\setcounter{page}{0}
\newpage
\pagenumbering{Roman}
\markboth{Selbstständigkeitserklärung}{}
\section*{Selbstständigkeitserklärung}
%\addcontentsline{toc}{section}{Selbstständigkeitserklärung} %Fügt das zum Inhaltsverzeichnis hinzu,
Wir versicheren hiermit, dass wir die Projektarbeit mit dem Thema: \glqq Entwicklung und Umsetzung einer intuitiven Steuerung für eine Roboterhand durch Erfassen
der Geste einer menschlichen Hand \grqq\ , selbst\-ständ\-ig verfasst und keine anderen als die angegebenen Quellen und Hilfsmittel benutzt haben.\vspace{2.5cm}
\begin{longtable}{p{5.5cm} p{3cm} p{6cm}}
 Frankfurt a. M., \today& & \\
 \cline{1-1} \cline{3-3}
 Ort, Datum& &Unterschrift (Abt)\\
\end{longtable}
\vspace{0.5cm}
\begin{longtable}{p{5.5cm} p{3cm} p{6cm}}
 Frankfurt a. M., \today& & \\
 \cline{1-1} \cline{3-3}
 Ort, Datum& &Unterschrift (Girke)\\
\end{longtable}
\addtocounter{table}{-1} % wird nicht mitgezählt bei der Tabellen Nr.
\newpage
%\markboth{Zusammenfassung}{}
%\section*{Zusammenfassung}
%\addcontentsline{toc}{section}{Zusammenfassung} %Fügt das zum Inhaltsverzeichnis hinzu
%\todo{brauchen wir das hier?}
%\newpage
\tableofcontents %Inhaltsverzeichis
\newpage
\phantomsection  %korregiert die links auf die Seite
\addcontentsline{toc}{section}{Abbildungsverzeichnis und Tabellenverzeichnis} %Fügt das Abbildungs zum Inhaltsverzeichnis hinzu
\listoffigures %abbildungsverzeichnis
\listoftables %tabellenverzeichnis
\newpage
\pagenumbering{arabic}
\section[Einleitung ]{Einleitung} \sectionmark{Einleitung}
Alles das \emph{so geschrieben ist, } ist von der Laboreinleitung\\
\emph{Einführung, Motivation}


\section{Stand der Technik}
\emph{Stand der Technik (Literatur/Patent-Recherche)}

Die Echtzeit-Erkennung von Handbewegungen ist für Steuerung von humanoiden Händen ist von Essenz. Die komplexen Bewegungsabläufe der menschlichen Hand lassen sich aufgrund der großen Anzahl an Fingersegmenten und Freiheitsgraden nur mit hohem Aufwand erfassen. Und eine Steuerung mittels Joysticks o.ä. ist ungeeignet.
Erste Versuche die Bewegungsabläufe der Hand aufzunehmen wurden mithilfe von in Handschuhen eingebauten Biegesensoren \cite{FlexSensor} und Lagesensoren durchgeführt.  Die zu dieser Zeit boomende Computerspielindustrie griff die Idee schnell auf und brachte den, technisch vereinfachten, PowerGlove \cite{PowerGlove} auf den Markt. Heute sind verschiedene Firmen im Markt die professionelle Systeme vertreiben wie CyberGlove Systems \cite{CyberGlove} oder Cobra Glove \cite{CobraGlove}. Diese bedienen sich meist der Erfassung der Fingerpositionen durch eine Kombination von mehreren an den Fingern angebrachten Inertial Measurment Units (IMUs) und Biegesensoren.

Handschuhe haben im allgemeinen einige Nachteile die sie mit sich bringen. Der an und Abziehvorgan ist umständlich, die Größe des Handschuhes muss stimmen, Desinfektionsmaßnahmen sind kompliziert.

Alternativ werden Handbewegungen auch mit Bewegungserkennungssystemen durch Marker und IR-Kamerasystemen aufgezeichnet. Über Triangolie die Position der einzellnen Markerpunkte berechnet. Hier ist die Firma VICON ein Vorreiter auf dem Markt.

Auch markerlose Kamerasysteme zur Bewegungserkennung existieren wie durch z.B. die Kinect Kamera ermöglicht.



\section{Mögliche Konzepte}
\emph{Experimental (Vorgehen/Methoden zur Konstruktion, Berechnung, Simulation)}
\subsection{Bautenzüge über Finger}
\subsection{Biegesensoren DMS}
Die heutzutage erhältlichen Biegesensoren sind wesenlicht preiswerter als die ersten ihrer Art. Sie basieren nicht mehr auf dem Prinzip eines Lichwellenleiters sondern auf der Änderung der Leitfähigkeit von Materialien durch deren Biegung. Somit ermöglichen sie eine Einfache Möglichkeit den Gebogenheitsgrad eines einzellnen Fingers zu bestimmen. Als schwieriger erweißt sich jedoch die Positionsbestimmung des Daumens. Dieser kann sich auch unabhängig seines Biegegrades auf dem unteren Sattelgelenk in zwei Freiheitsgraden bewegen. Ein Begesensor reicht nicht um z.B. den Pinzettengriff zwischen Daumen und Zeigefinger und zwischen Daumen und Mittelfinger zu unterschieden.

\subsection{Image Processing}
Mithilfe des Einsatzes von moderner Bildverarbeitung und künstlicher Intelligenz lassen sich viele Probleme der vorherigen Methoden vermeiden. So sind allen vorran die benötigten Investitionen nahe Null. Lediglich ein PC sowei eine passende Webcam sind bereits aussreichen um die Positionserkennung zu ermöglichen. Ausserdem fallen umständliches an und abziehen eines sensiblne Handschuhes so wie der damit verbundene verkabelungs Aufwand weg.

\section{Umsetzung des Konzepts}
\emph{Ergebnisse (CAD-Modelle, Funktionsmuster, Messdaten, etc.)}\\
Konzept Auswahl
\section{Fazit}
\emph{Diskussion (Interpretation und Beurteilung der Ergebnisse)}
\section{Ausblick}
\emph{Zusammenfassung und Ausblick (Vorschläge für weiterführende Arbeiten)}
\newpage
\pagenumbering{Alph}
\phantomsection 
\addcontentsline{toc}{section}{Literaturverzeichnis} %Fügt das Literaturverzn soeichnis zum Inhaltsverzeichnis hinzu
\printbibliography  % Literaturliste ausgeben.
\newpage
\section*{Anhang} \sectionmark{Anhang}
\addcontentsline{toc}{section}{Anhang} %Fügt den Anhang zum Inhaltsverzeichnis hinzu
\begin{enumerate}
	\item Code für Ansteuerung Klemmgreifer \& Magnetgreifer
\end{enumerate}
\end{document}
